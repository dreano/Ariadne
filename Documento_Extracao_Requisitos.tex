\documentclass[12pt,twoside,a4paper]{article}
\usepackage[brazil]{babel}
\usepackage[utf8]{inputenc}
\usepackage[T1]{fontenc}
\usepackage{timbre-ic}
\usepackage{booktabs}
\usepackage[table]{xcolor}
\usepackage{url}
\usepackage{array}
\usepackage{graphicx}
\usepackage{pifont}
\usepackage{float}
\usepackage{multirow}
\usepackage{indentfirst}

\begin{document}

\begin{titlepage}
\begin{center}
{\large Universidade Estadual de Campinas}\\[0.2cm]
{\large Instituto de Computação}\\[0.2cm]
%{\large Nome do departamento}\\[0.2cm]
{\large Ciência da Computação/Engenharia da Computação}\\[0.2cm]
{\large MC426A/MC436A}\\[5.1cm]
{\bf \huge Documento de Especificação de Requisitos de Software}\\[5.1cm]
\end{center}
{\large Alunos:\\William Henry Gates III RA123456\\ Steven Paul Jobs RA 654321}\\[0.7cm]
{\large Professora: Ariadne Maria B. Rizzoni Carvalho}\\[3.1cm]
\begin{center}
{\large Campinas}\\[0.2cm]
{\large 2014}
\end{center}
\end{titlepage}

%%\vskip 15mm

\tableofcontents

\newpage

\section{Introdução}

Este documento apresenta a especificação de requisitos para a informatização do portal de xxxx.  O portal de venda xxx uuuu...

\textit{Descrever o objetivo do sistema, os ganhos esperados para a empresa e para os usuários}

\textit{\textbf{Importante:} elaborar diagrama mostrando o sistema e a interação com outros sistemas/Serviços. Pode ser elaborado desenhado de forma manual e “colado” no documento} 


\section{Glossário}
\textit{Elaborar uma lista de termos que serão utilizados no documento, definindo o significado no contexto do projeto}

\textbf{Requisitos Funcionais} - Funcionalidades que se espera que o sistema disponibilize, de uma forma completa e consistente.

\textbf{Requisitos Não-Funcionais} - são os requisitos relacionados ao uso da aplicação em termos de desempenho, usabilidade, confiabilidade, segurança, disponibilidade, manutenibilidade e tecnologias envolvidas.
 
\textbf{Sistema} -  

\textbf{SLA} - 
 
\textbf{SSL} - 
 
\textbf{XXXXX}

\textit{Também poderia ser elaborado em forma de tabela}

\section{Definição dos requisitos de usuário}
\subsection{Requisitos Funcionais}

\textbf{RF01.} O portal deve oferecer a busca de passagens aéreas.

\underline{Informações:} Local de origem, local de destino, data de partida, data de volta, quantidade  de adultos, quantidade de crianças, idade das crianças.

\underline{Regras:} O sistema deve permitir que o visitante ou o cliente do portal possam executar a busca fornecendo as informações obrigatórias e, caso haja resultado para a consulta, o portal deve disponibilizar as passagens aéreas com seus preços, as companhias disponíveis, escala e conexões (se houver), origem, destino e horário. Caso não haja disponibilidade, deve ser informado ao cliente que não houve resultado para a consulta. A obtenção dos dados de passagens de avião deve ser efetivada através de uma interface de integração entre o portal e as companhias aéreas.

\begin{itemize}

\item O portal deve validar se as informações sobre local de origem, local de destino, data de partida, data da volta e número de pessoas estão preenchidos.
\item A data de partida deve ser maior ou igual à data corrente.
\item A data de volta deve ser maior ou igual à data de partida.
\newpage
\item O número de adultos ou o número de crianças deve ser maior que zero. Por padrão, o portal deve preencher o número de adultos com 1(um) .
\item Caso o número de crianças seja maior que 0 (zero), a idade deve ser preenchida.
\item Caso alguma das informações acima não cumpra essas regras, o portal deve informar o usuário a respeito do erro de preenchimento. 
\item Caso o valor das passagens para crianças seja diferente do valor das passagens para adultos, deve ser apresentada qual a regra utilizada, por exemplo: "Crianças acima de 6 (seis) anos devem pagar valor integral".

\end{itemize}

\textbf{RF02. }

\textbf{RF03.} 

\textbf{(Etc...)}

 


\subsection{Requisitos Não Funcionais}

\textbf{RNF01.} O sistema deverá armazenar as informações de senha e dados bancários utilizando criptografia

\textbf{RNF02.} O sistema deverá ser independente de plataforma hardware e....

\section{Evolução do Sistema}

O sistema baseado em plataforma WEB estará preparado para integrar as seguintes funcionalidades:
Disponibilizar acesso através de plataforma móvel (smartphone);
XXX



\section{Diagrama de Hierarquia de Pontos de vista (HPV)}

\textit{Elaborar o diagrama de pontos de Vista.  Use qualquer ferramenta e “cole” o resultado}

\newpage

\section{Anexo}
\subsection{Tabelas VORD}

\textit{Isto é um exemplo de  pontos de vistas   identificados no diagrama}

\begin{table}[H]
\label{tab:VORD01}
\begin{small}
\begin{center}
    \begin{tabular}{ | l | l |}
    \hline
\textbf{Referência} & {Administrador do sistema}\\ \hline
\textbf{Atributos} & Login, senha, matrícula, nome e email \\\hline
\multirow{4}{*}{\textbf{Eventos}} & Administrar base de dados referente às informações disponibilizadas no sistema \\ & (parcerias, itens dos pacotes, convênios) \\ & Cadastrar novos usuários (administradores e operadores) do sistema \\ & Gerenciar logs\\\hline
\multirow{3}{*}{\textbf{Serviços}} & Cadastro, Alteração e Remoção de operador do sistema\\ & Cadastro, Alteração e Remoção de administradores do sistema\\& Consulta de logs\\\hline
\multirow{2}{*}{\textbf{Subpontos de vista}} & Operador do sistema \\ &
Usuário do sistema \\\hline
    \end{tabular}
\end{center}    
\end{small}
\end{table}

\begin{table}[H]
\label{tab:VORD02}
\begin{small}
\begin{center}
    \begin{tabular}{ | l | l |}
    \hline
\textbf{Referência} & {Operador do sistema}\\ \hline
\textbf{Atributos} & Login, senha, matrícula, nome e email \\\hline
\multirow{3}{*}{\textbf{Eventos}} & Cadastrar, Alterar e Remover passeios, hotéis, guia turístico, operadoras de \\& cartão de crédito, bancos, companhias aéreas e marítimas\\ & Visualizar e alterar a situação das reservas dos usuários. \\\hline
\multirow{3}{*}{\textbf{Serviços}} & Cadastrar, Alterar e Remover passeios\\ & Cadastrar, Alterar e Remover companhias marítimas \\& Visualizar e alterar a situação das reservas dos usuários\\\hline
\textbf{Subpontos de vista} & Usuário do sistema\\\hline
\textbf{Provedor} & Nenhum\\\hline
    \end{tabular}
\end{center}    
\end{small}
\end{table}

\section{Bibliografia}

[1]  Rizzoni, Ariadne M. B. e Chiossi, Thelma C. dos Santos. Introdução à Engenharia de Software. Editora da Unicamp, 2001.\\

[2]  Sommerville, Ian. Software Engineering. Pearson, 2010.\\

[3]  Lobatoxxxxxxxxxxxxxxxx. Disponível em <www.xxxxxxxxxx.com/ssl.ppt>. Acesso em 01/03/2011.

\newpage
\section{Técnica de levantamento utilizada}

\textit{Houve entrevistas? Exemplos de questões. Veja exemplo:}

Depois que o cliente fechou o pacote, ele poderá ter acesso às informações do pacote?

Sim, pode visualizar e até mesmo imprimir.


\textit{Brainstorming? Como foi conduzido}

\end{document}
