\documentclass[12pt,twoside,a4paper]{article}
\usepackage[brazil]{babel}
\usepackage[utf8]{inputenc}
\usepackage[T1]{fontenc}
\usepackage{timbre-ic}
\usepackage{booktabs}
\usepackage[table]{xcolor}
\usepackage{url}
\usepackage{array}
\usepackage{graphicx}
\usepackage{pifont}
\usepackage{float}
\usepackage{multirow}


\begin{document}

%%\vskip 15mm

\begin{center} 
\textbf{MO426/MO436 - Requisitos de Software e Modelos de Especificação}

\textbf{Aula Prática - Levantamento de Requisitos do \textit{e-commerce}
CompreFacil.com}

\vskip 20mm

\textbf{Atividade}

\end{center}

\begin{enumerate}
\item \textbf{Levantamento dos requisitos do Sistema CompreFacil.com}

	\begin{itemize}

		\item Identificar e priorizar os usuários e os requisitos funcionais e não funcionais
		\item Técnica: \textit{BrainStorm} 

	\end{itemize}
	
\item \textbf{Descrição dos requisitos}

	\begin{itemize}
	
	\item Objetivo: Descrever os requisitos priorizados, os dados manipulados e as regras de negócio identificadas.
	\item Exemplo: 

	a)Criar conta única de cliente no Portal CompreFacil.com. O Portal
	deve prover uma funcionalidade que permita a criação de uma conta única,
	através da qual o usuário terá acesso a todas as demais funções do sistema. Informações: e-mail, nome, cpf, nome de usuário, senha, telefone e endereço. Regras: (a) 	um e-mail deve ser enviado para o e-mail informado formalizando a criação da conta. (b) O endereço deve ser um endereço válido na lista dos correios. (c) o nome do usuário deverá ser único.
	
	\end{itemize}
	
\item Identificação dos serviços e pontos de vista

	\begin{itemize}
	
	\item Objetivo:
	Identificar
	
	(A) os pontos de vista
	
	(B) os serviços
	
	(c) associar lista de serviços aos pontos de vista 
	
	\item Exemplo:
	
	\end{itemize}

\end{enumerate}

\begin{table}[H]
\label{tab:printers}
\begin{small}
\begin{center}
    \begin{tabular}{ | l | l |}
    \hline
Referência & \textbf{Cliente}\\ \hline
Atributos & Código; Nome; E-mail; Telefone; Nome do usuário; Senha;
Endereço \\\hline
\multirow{3}{*}{Eventos} & Confirmar cadastro de usuario\\ & Verificar se foram enviados e-maill com comunicações emitidas pelo portal \\ & Receber e-mail notificando que a criação da conta foi aceita\\\hline
\multirow{6}{*}{Serviços} & Logar\\ & Criar conta\\& Editar conta\\& Comprar produtos\\& Pesquisar produtos\\& Etc...\\\hline
\multirow{2}{*}{Subpontos de vista} & Visitante \\ &
Operador\\\hline
    \end{tabular}
\end{center}    
\end{small}
\end{table}

\end{document}
