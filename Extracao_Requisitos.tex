\documentclass[12pt,twoside,a4paper]{article}
\usepackage[brazil]{babel}
\usepackage[utf8]{inputenc}
\usepackage[T1]{fontenc}
\usepackage{timbre-ic}
\usepackage{booktabs}
\usepackage[table]{xcolor}
\usepackage{url}
\usepackage{array}
\usepackage{graphicx}
\usepackage{pifont}
\usepackage{float}


\begin{document}

\vskip 15mm

\begin{center} 
\textbf{MO426/MO436 - Requisitos de Software e Modelos de Especificação}

\textbf{Aula Prática - Levantamento de Requisitos do \textit{e-commerce}
CompreFacil.com}

\end{center}

\vskip 5mm

\vskip 20mm

\textbf{Atividade}

\begin{enumerate}
\item \textbf{Levantamento dos requisitos do Sistema CompreFacil.com}

	\begin{itemize}

		\item Identificar e priorizar os usuários e os requisitos funcionais e não funcionais
		\item Técnica: \textit{BrainStorm} 

	\end{itemize}
	
\item \textbf{Descrição dos requisitos}

	\begin{itemize}
	
	\item Objetivo: Descrever os requisitos priorizados, os dados manipulados e as regras de negócio identificadas.
	\item Exemplo: 

	a)Criar conta única de cliente no Portal CompreFacil.com. O Portal
	deve prover uma funcionalidade que permita a criação de uma conta única,
	através da qual o usuário terá acesso a todas as demais funções do sistema. Informações: e-mail, nome, cpf, nome de usuário, senha, telefone e endereço. Regras: (a) 	um e-mail deve ser enviado para o e-mail informado formalizando a criação da conta. (b) O endereço deve ser um endereço válido na lista dos correios. (c) o nome do usuário deverá ser único.
	
	\end{itemize}

\end{enumerate}
% resetando configs de layout
\newpage
\pagestyle{plain}
\headheight 0.0cm
\headsep 0.0cm
\footskip 2.2cm

\section{Exercício}
\label{sec:01}

\textbf{Resposta:} 

\begin{table}[H]
\label{tab:printers}
\begin{small}
\begin{center}
    \begin{tabular}{ | c | c | c | c | c |}
    \hline
Descrição & Endereço/Prefixo &  & Descrição & Endereço/Prefixo\\ \hline
Infraestrutura rot. & 2001:db8::/48 &  & Rede 7 (R7) & 2001:db8:3:100::/56\\\hline
Gestão e Monit. & 2001:db8:1::/48 &  & Rede do Host1 & 2001:db8:2::/64\\\hline
Rede 1 (R1) & 2001:db8:2::/48 &  & Rede do Host2 & 2001:db8:3::/64\\\hline
Rede 2 (R2) & 2001:db8:3::/48 &  & Rede do Host3 & 2001:db8:3:1::/64\\\hline
Rede 3 (R3) & 2001:db8:4::/48 &  & Host1 & 2001:0db8:0002:0000:0000:0000:0000:0001/64\\\hline
Rede 4 (R4) & 2001:db8:2::/56 &  & Host2 & 2001:0db8:0003:0000:0000:0000:0000:0002/64\\\hline
Rede 5 (R5) & 2001:db8:3::/56 &  & Host3 & 2001:0db8:0003:0000:0000:0000:0000:0003/64\\\hline
Rede 6 (R6) & 2001:db8:2:100::/56 &  &  & \\
\hline
    \end{tabular}
\end{center}    
\end{small}
\end{table}

\end{document}
